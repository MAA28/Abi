\documentclass[a4paper]{article}
\usepackage{hyperref}
\usepackage{hyphenat}

\title{Exposé\\Imperialismus als höchstes Stadium des Imperialismus: Überprüfen der Thesen der Kapitel ``Konzentration der Produktion und Monopole`` in der Bundesrepublik Deutschland zu Beginn des 21. Jahrhunderts}
\author{Malte Arved Aschenbach}
\begin{document}
    \maketitle
    Als zu Beginn des 20. Jahrhunderts der Kapitalismus durch seine finale Entwicklung gang und aus der Phase der freien Konkurrenz in die Phase der Monopole und des Imperialismus überging, verfasste W.I. Lenin in Zürich in 1916 eines der wichtigsten Werke des wissenschaftlichen Sozialismus in dem er diese neu Phase analysierte und ihre Konsequenzen erkannte. Dieses Werk trägt den Namen ``Imperialismus als höchstes Stadium des Kapitalismus``.
    
    Da die Verfassung dieses Buches mittlerweile über 100 Jahre her ist, stellt sich die offensichtliche Frage, ob man von dieser Analyse heute noch behaupten kann relevant zu sein. Um die Beantwortung dieser Frage zu beginnenen werde ich überprüfen, ob die in Kapitel ``Konzentration der Produktion und Monopole`` getroffenen Schlüsse heute in vergleichbarer Form heute getroffen werden könnten.
    
    Hierzu ist es zu erst notwendig die Analyse des Kapitels zu beschreiben. Um dieses Kapitel in den richtigen Kontext zu setzen halte ich es außerdem für wichtig die die Bedingungen in denen Lenin das Buch geschrieben hat und die Rezepeztion von bürgerlichen und marxistischen Ökonomen anzuschauen.

    %(https://de.statista.com/statistik/daten/studie/173884/umfrage/zahl-der- unternehmen-in-den-einzelnen-marktbereichen-des-energiemarktes/) 
    Für die Überprüfung der heutigen Situation werde ich mich einerseits auf die ökonomische Statistiken beziehen und andererseits auf die Beobachtungen der verschiedener Kommunist*innen stützen.
    
    Beispielsweise hat man durchschnittlich die Wahl zwischen 142 Stromlieferanten (Monitoringbericht der Bundesnetzagentur und des Bundeskartellamts, S. 11), da E.ON einige dieser Lieferanten besitzt und andere Verdrängt hat, hat dieser Konzern einen Marktanteil von 20-30 Prozent (\href{https://www.welt.de/wirtschaft/article189294857Strommarkt-Angst-vor-Dominanz-von-E-on-und-RWE-waechst.html}{link}) (Hintergrundpapier, Februar 2019 ``Weniger Wettbewerb \& höhere Strompreise. Warum der RWE-Eon-Deal zu Lasten der Kunden geht.``, S. 4) darduch ist E.ON in vielen Situationen außerhalb der freien Konkurrenz. Andere Beispiele für die Konzertration der Produktion wären die Pharmaindustrie und die Automobilhersteller. 
    
    Im Bezug auf die praktische Anwendung Lenins Theorie, könnte ich das Zukunftspapier der Sozialistischen Deutschen Arbeiterjugend zitieren: In dem Kapitel ``Imperialismus - der Kapitalius unserer Zeit'' schreibt die SDAJ ``Die oberen Zehntausend beherrschen die deutsche Wirtschaft, viele kleine und mittlere Unternehmen sind völlig abhängig von Großkonzernen und -banken. Durch ein Netz aus Verbänden, Stiftungen, offiziellen und inoffiziellen Gremien sichern sich die Großkonzerne den Einfluss auf den Staatsapparat''.

    Doch wird diese Facharbeit unmöglich alle Thesen untersuchen können. Daher wäre es sinnvoll in dem Abschluss zu erwähnen, dass man in zukunftigen Arbeiten auch die vier weiteren Merkmale des Imperialismus nach Lenin überprüfen sollte um die Analyse zu vervollständigen. 
    

    
\end{document}
